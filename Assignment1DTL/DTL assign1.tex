\documentclass{report}
\title{Sections and Chapters}
\author{Amaan Jamadar}
\date{\today}
\begin{document}
\maketitle
\tableofcontents
\chapter{An Introduction to Lua\TeX}

\section{What is it—and what makes it so different?}
Lua\TeX{} is a \textit{toolkit}—it contains sophisticated software tools and components with which you can construct (typeset) a wide range of documents. The sub-title of this article also poses two questions about Lua\TeX: What is it—and what makes it so different? The answer to “What is it?” may seem obvious: “It’s a \TeX{} typesetting engine!” Indeed it is, but a broader view, and one to which this author subscribes, is that Lua\TeX{} is an extremely versatile \TeX-based document construction and engineering system.

\subsection{Explaining Lua\TeX: Where to start?}
The goal of this first article on Lua\TeX{} is to offer a context for understanding what this TeX engine provides and why/how its design enables users to build/design/create a wide range of solutions to complex typesetting and design problems—perhaps also offering some degree of “future proofing” 

\chapter{Lua\TeX: Background and history}
\section{Introduction}
Lua\TeX{} is, in \TeX{} terms, “the new kid on the block” despite having been in active development for over 10 years.

\subsection{Lua\TeX: Opening up \TeX’s “black box”}
Knuth’s original \TeX{} program is the common ancestor of all modern \TeX{} engines in use today and Lua\TeX{} is, in effect, the latest evolutionary step: derived from the pdf\TeX{} program but with the addition of some powerful software components which bring a great deal of extra functionality.
\end{document}